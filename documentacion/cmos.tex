%!TEX encoding = UTF-8 Unicode%
% Este es el trabajo de Gabinete expedido el Mayo de 2010.
% Tema: CMOS Setup
% Integrantes:
%	* Kenny Meyer
%	* Ana Benitez
%	* Maria Monsserat Silva
%	* Fabiola Ruiz Diaz
%
\documentclass[12pt,oneside,a4paper]{article}
%% Use the Arial font:
\usepackage[T1]{fontenc}
\usepackage[scaled]{times}
\renewcommand*\familydefault{\sfdefault}

%%% Provide underlining:
%\usepackage{ulem}

%% EN: Set language to Spanish
%% ES: Definir idioma como Castellano
\usepackage[spanish]{babel}
\selectlanguage{spanish}

\usepackage[pdftex]{graphicx}
\usepackage[utf8]{inputenc}
\usepackage{url} % para URLs en la bibliografía

%% EN: Set page geometry
%% ES: Definir la geometria de la pagina
\usepackage[a4paper]{geometry}
\geometry{top=4.0cm, bottom=3.0cm, left=2.01cm, right=2.01cm}

%% Make a fancy header/footer
\usepackage{fancyhdr}
\pagestyle{fancy}
%% Define custom header:
\fancyhead{}
\fancyhead[CO,CE]{\bfseries{CMOS Setup}}
%% Define custom footer:
\fancyfoot{}
\fancyfoot[C]{\thepage}

\begin{document}
\title{CMOS Setup}
\author{Ana Benitez (\texttt{ana\_benitez\_py@hotmail.com}), \\
		Maria Monserrat Silva(\texttt{monse\_14\_fob@hotmail.es}), \\
		Fabiola Ruiz Leiva(\texttt{fabiola.ruiz.leiva@gmail.com}), \\ 
		Kenny Meyer (\texttt{knny.myer@gmail.com})}
\date{Mayo 2010}
\maketitle
\clearpage

% Factores que tener en cuenta:
%  * La documentacion sera leida por los companeros
%  * Tiene que ser comprensiva y completa.
%
\setcounter{tocdepth}{4}
\tableofcontents

\newpage

\section{Introducción}{\label{sec:introduccion}}

Cuando encendemos la computadora, el sistema operativo se encuentra o bien, en
el disco duro o bien en un disquete; sin embargo, si se supone que es el
sistema operativo el que debe dar soporte para estos dispositivos, {\em ¿cómo podría
hacerlo si aún no está cargado en memoria?}

Lo que es más:
\begin{itemize}
	\item ¿Cómo sabe la computadora que tiene un disco duro (o varios)?
	\item ¿Y la disquetera?  
	\item ¿Cómo y dónde guarda esos datos, junto con el tipo de memoria y cache?
	\item ¿O algo tan sencillo pero importante como la fecha y la hora? 
\end{itemize}

Por ende, la unidad y programa que se encarga de todo esto es la CMOS Setup o BIOS.
Resulta evidente que la BIOS debe poderse modificar para alterar estos datos
(al añadir un disco duro o cambiar al horario de verano, por ejemplo); por ello
las BIOS se implementan en memoria. \\
Pero además debe mantenerse cuando apaguemos nuestra pc, porque no tendría
sentido tener que introducir todos los datos en cada arranque; por eso se usan
memorias especiales, que no se borran al apagar el ordenador: memorias tipo
CMOS (no volatil), por lo que muchas veces el programa que modifica la BIOS se
denomina {\bf CMOS Setup}.

\newpage

	\section{Conceptos}{\label{sec:conceptos}}

		\subsection{La RAM}{\label{sec:conceptos/RAM}}

		La memoría de acceso aleatorio (en {\bfseries inglés Random Access Memory} cuyo
		acrónimo es {\bfseries RAM}) es la memoría donde el procesador recibe las
		instrucciones y guarda los resultados. Es el área de trabajo paara la mayor
		parte del software de un computador.

		\subsection{La ROM}{\label{sec:conceptos/ROM}}

		Memoría de solo lectura ({\bfseries Read-only memory}) es una
		clase de medio de almacenamiento usado en los ordenadores y otros dispositivos
		electrónicos. Los
		datos en la ROM no se pueden modificar al menos no de una manera rápida o fácil
		y se usa principalmente para contener el Firmware.

		\subsection{El Firmware}{\label{sec:conceptos/firmware}}

		Firmware o programación en firme, es un bloque de instrucciones de
		programa para propositos específicos grabada en una memoría de tipo no volátil
		( ROM, EPROM, Flash, ...) que establece la lógica de más bajo nivel que
		controla los circuitos electrónicos de un dispositivo de cualquier tipo.

		\subsection{El CMOS}{\label{sec:conceptos/CMOS}}

		{\bf CMOS} es la contracción de {\em Complementary Metal Oxide Semiconductor}, o
		Tecnología Metal Óxido Semiconductor Complementario.
		
		Es una tecnología utilizada para fabricar circuitos integrados (chips), que
		para el caso es un memoria del tipo ROM, en la actualidad FLASH ROM (se
		borran electricamente) donde se guarda información básica del equipo como el
		BIOS y el SETUP. 
		
		Esta memoria tiene una porción que funciona como RAM, donde se guarda la
		configuración que el usuario le da al equipo y algunas configuraciones
		establecidas de fábrica. \\ 
		Otra información que se guarda en el {\bf CMOS} y que se accede mediante el
		SETUP es la fecha y la hora del sistema, y para que la misma se mantenga
		actualizada debe estar siempre alimentada, es por esto que en toda
		computadora encontraremos una pila para alimentar dicho circuito.

		\subsection{La BIOS}{\label{sec:conceptos/BIOS}}

		BIOS es la contracción de Basic Input Output System, o Sistema Básico de
		Entrada – Salida. 
		Es un programa muy básico, normalmente programado en lenguaje ensamblador,
		cuya misión es la de arrancar o posibilitar el “Booteo” de la computadora.
		A pesar de tratarse de un programa sumamente básico resulta totalmente
		indispensable, ya que sin el BIOS sería imposible que una computadora pudiera
		iniciar.

	\newpage

\section{El BIOS}{\label{sec:bios}}

Es indispensable conocer el concepto y las funciones que cumple el BIOS antes de tratar el CMOS Setup.

El concepto del BIOS:
\begin{quotation}

	{\em BIOS es la contracción de Basic Input Output System, o Sistema Básico de
	Entrada – Salida. 
	Es un programa muy básico, normalmente programado en lenguaje ensamblador,
	cuya misión es la de arrancar o posibilitar el “Booteo” de la computadora.
	A pesar de tratarse de un programa sumamente básico resulta totalmente
	indispensable, ya que sin el BIOS sería imposible que una computadora pudiera
	iniciar.}

\end{quotation}

Esta seccion tratara todo lo referente al BIOS.

	\subsection{El arrancar de la Computadora}{\label{sec:bios/arranque}}

	Una vez que presionamos el botón power de nuestra computadora, el BIOS
	guardado en la CMOS se copia en la RAM y es ejecutado por el
	microprocesador, aunque en las motherboards actuales también puede ser
	ejecutado directamente desde la CMOS.

	Cuando la computadora arranca el BIOS testea y activa una serie de
	elementos de hardware del sistema, tales como el teclado, monitor y
	unidades de almacenamiento, efectúa un proceso de comprobación de los
	mismos denominado POST, Power On Self Test o programa de autotesteo, carga
	una serie de configuraciones establecidas, tanto por el usuario como por el
	propio BIOS, busca el sistema operativo entre los distintos medios de
	almacenamiento presentes, carga ste en la RAM, lo hace residente y luego le
	transfiere el control de la computadora al mismo.

	A partir de este punto acaba de trabajar ya que todo el control pasa al sistema
	operativo.

	\subsection{Que hace el BIOS?}{\label{sec:bios/que-hace-la-bios}}
	\subsection{Configuracion de la BIOS}{\label{sec:bios/configuracion-de-bios}}
	La mayoría de los BIOS tienen un programa de configuración que permite modificar la configuración básica del sistema. Este tipo de información se almacena en una memoria auto-alimentada (por medio de una batería), para que la información permanezca almacenada incluso si el ordenador se encuentra apagado (la memoria RAM se reinicia cada vez que se inicia el sistema).
	Cada equipo cuenta con varios BIOS: 
	El BIOS de la placa madre 
	El BIOS que controla el teclado 
	El BIOS de la tarjeta de video 
	y eventualmente, 
	El BIOS para controladoras SCSI, que se utiliza para iniciar desde un dispositivo SCSI, el que luego se comunica con el DOS, sin que se necesite un controlador adicional. 
	(El BIOS de la tarjeta de red para iniciar desde una red) 
	Cuando se enciende el ordenador, el BIOS muestra una mensaje de copyright en pantalla, luego realiza los diagnósticos y pruebas pertinentes a la inicialización. Luego de completadas las pruebas, el BIOS muestra un mensaje en el que se invita al usuario a que presione una o más teclas para ingresar a la configuración del BIOS. 
	Según la marca de BIOS, puede tratarse de la tecla F2, de la tecla F10, o bien de la tecla Supr, o alguna de las siguientes secuencias de teclas: 
	Ctrl+Alt+S 
	Ctrl+Alt+Esc 
	Ctrl+Alt+Ins 
	En los BIOS Award, se muestran los siguientes mensajes durante el POST:
	PARA INGRESAR A LA CONFIGURACIÓN ANTES DE REINICIAR PRESIONE CTRL-ALT-ESC O LA TECLA DEL

	\subsection{Solucion de problemas}{\label{sec:bios/solcion-de-problemas}}
		\subsubsection{Pitidos de la BIOS (Award)}{\label{sec:bios/tonos-de-la-bios}}

		En esta seccion cubriremos los significados de los pitidos de los BIOS {\em Award}. \\
		En la mayoría de los pitidos se les acompaña un mensaje de error. 

			\paragraph{Tono ininterrumpido}:
			
			Fallo en el suministro elctrico. Revisamos las conexiones y la fuente
			de alimentación.  Tonos cortos constantes: Sobrecarga elctrica, chips
			defectuosos, placa mal.

			\paragraph{1 largo}:

			Si aparece esto en la pantalla “RAM Refresh Failure”, significa que los
			diferentes componentes encargados del refresco de la memoria RAM fallan
			o no están presentes. Cambiar de banco la memoria y comprobar los
			jumpers de buses. 

			\paragraph{1 largo y 1 corto}: 

			El código de la BIOS esta corrupto o defectuoso, probaremos a flasear o
			reemplazamos el chip de la BIOS sino podemos cambiamos de placa. 

			\paragraph{1 largo y dos cortos}:
			
			No da señal de imagen, se trata de que nuestra tarjeta de vídeo esta
			estropeada, probaremos a pincharla en otro slot o probaremos otra
			tarjeta gráfica. 

			\paragraph{1 largo y 2 cortos}:

			Si aparece por pantalla este mensaje: “No video card found”, este error
			solo es aplicable a placas base con tarjetas de vídeo integradas. Fallo
			en la tarjeta gráfica, probaremos a deshabilitarla y pincharemos una
			nueva en cualquier slot libre o cambiaremos la placa madre. 

			\paragraph{1 largo y 3 cortos}:

			Si aparece este mensaje por pantalla “No monitor connected” Idem que el
			anterior. 

			\paragraph{1 largo y varios cortos}: 

			Mensaje de error. “Video related failure”. Lo mismo que antes. Cada
			fabricante implanta un código de error según el tipo de tarjeta de
			video y los parámetros de cada BIOS.

			\paragraph{2 largos y 1 corto}:
			
			Fallo en la sincronización de las imágenes. Cargaremos por defecto los
			valores de la BIOS e intentaremos reiniciar. Si persiste nuestra
			tarjeta gráfica o placa madre están estropeadas. 

			\paragraph{2 cortos}:
			
			Vemos en la pantalla este error: “Parity Error”. Se trata de un error
			en la configuración de la BIOS al no soportar la paridad de memoria, la
			deshabilitamos en al BIOS. 

			\paragraph{3 cortos}:
			
			Vemos en la pantalla este error. Base 64 Kb “Memory Failure”, significa
			que la BIOS al intentar leer los primeros 64Kbytes de memoria RAM
			dieron error. Cambiamos la RAM instalada por otra. 

			\paragraph{4 cortos}:
			
			Mensaje de error; “Timer not operational”. El reloj de la propia placa
			base esta estropeado, no hay más solución que cambiar la placa. No
			confundir con “CMOS cheksum error” una cosa es la pila y otra el
			contador o reloj de la placa base. 

			\paragraph{5 cortos}:
			
			Mensaje por pantalla ``Processor Error'' significa que la CPU ha generado
			un error porque el procesador o la memoria de vídeo están bloqueados. 

			\paragraph{6 cortos}:

			Mensaje de error: ``8042 - Gate A20 Failure'', muy mítico este error.
			El controlador o procesador del teclado (8042) puede estar en mal
			estado. 
			La BIOS no puede conmutar en modo protegido. Este error se
			suele dar cuando se conecta/desconecta el teclado con el ordenador
			encendido. 

			\paragraph{7 cortos}:
			
			Mensaje de error: “Processor Exception / Interrupt Error” Descripción. La CPU ha generado una interrupción excepcional o el modo virtual del procesador está activo. Procesador a punto de morirse. 

			\paragraph{8 cortos}:
			
			Mensaje de error: ``Display Memory Read / Write error''. La tarjeta de video esta estropeada, procedemos a cambiarla. 

			\paragraph{9 cortos}:
			
			Mensaje de error: ``ROM Checksum Error''; el valor del checksum (conteo de la memoria) de la RA

		\newpage

		\subsubsection{Errores en pantalla}\label{sub:errores en pantalla}
		
		Los siguientes errores de pantalla son errores generales o bien, no dependen de marca y modelo de la BIOS:

			\paragraph{BIOS ROM checksum error – system halted}

			El código de control de la BIOS es incorrecto, lo que indica que puede
			estar corrupta. En caso de reiniciar y repetir el mensaje, tendremos
			que reemplazar al BIOS.

			\paragraph{CMOS battery failed}

			La pila de la placa base que alimenta la memoria CMOS ha dejado de
			suministrar corriente. Es necesario cambiar la pila inmediatamente.

			\paragraph{CMOS checksum error – Defaults loaded}

			El código de control de la CMOS no es correcto, por lo que se
			procede a cargar los parámetros de la BIOS por defecto. Este error
			se produce porque la información almacenada en la CMOS es
			incorrecta, lo que puede indicar que la pila está empezando a
			fallar.

			\paragraph{Display switch is set incorrectly}

			El tipo de pantalla especificada en la BIOS es incorrecta. Esto puede ocurrir si hemos seleccionado la existencia de un adaptador monocromo cuando tenemos uno en color, o al contrario. Bastará con poner bien este parámetro para solucionar el problema.

			\paragraph{Floppy disk(s) Fail} 
			
			(code 40/38/48 dependiendo de la antigüedad de la bios)

			Disquetera mal conectada, verificamos todos los cables de conexión.

			\paragraph{Hard disk install failure}

			La BIOS no es capaz de inicializar o encontrar el disco duro de
			manera correcta. Debemos estar seguros de que todos de que todos
			los discos se encuentren bien conectados y correctamente
			configurados.

			\paragraph{Keyboard error or no keyboard present}

			No es posible inicializar el teclado. Puede ser debido a que no se
			encuentre conectado, este estropeado e incluso porque mantenemos
			pulsada alguna tecla durante el proceso de arranque.

			\paragraph{Keyboard error is locked out – Unlock the key}

			Este mensaje solo aparece en muy pocas BIOS, cuando alguna tecla ha
			quedado presionada. 

			\paragraph{Memory Test Fail}

			El chequeo de memoria RAM ha fallado debido probablemente, a
			errores en los módulos de memoria. En caso de que nos aparezca este
			mensaje, hemos de tener mucha precaución con el equipo, se puede
			volver inestable y tener prdidas de datos. Solución, comprobar que
			banco de memoria está mal, y sustituirlo inmediatamente. 

			\paragraph{Override enabled – Defaults loaded}

			Si el sistema no puede iniciarse con los valores almacenados en la
			CMOS, la BIOS puede optar por sustituir estos por otros genricos
			diseñados para que todo funcione de manera estable, aunque sin
			obtener las mayores prestaciones. 

			\paragraph{Primary master hard diskfail}

			El proceso de arranque ha detectado un fallo al iniciar el disco
			colocado como maestro en el controlador IDE primario. Para
			solucionar comprobaremos las conexiones del disco y la
			configuración de la BIOS. 

	\newpage

\section{El CMOS Setup}{\label{sec:cmossetup}}
	\subsection{Resetear el CMOS Setup}{\label{sec:cmossetup/resetear-el-cmos-setup}}

	\newpage

\begin{thebibliography}{9}{\label{sec:bibliografia}}

	\bibitem{WIC}
		What is CMOS?,
		\url{http://www.wisegeek.com/what-is-cmos.htm}

	\bibitem{UnidadIIBIOSExposicion.ppt}
		% CORREGIR!
		Exposicion de BIOS de la Universidad BLABLA

	\bibitem{BIOS}
		``BIOS'' - {\em Basic Input-Output System}
		\url{http://es.wikipedia.org/wiki/BIOS}

	\bibitem{CIL}
		Dan Clein, ``CMOS IC LAYOUT Concepts, Methodologies, and Tools'', pp. 22 - 26, 2000

	\bibitem{MDB}
		Mesmer, ``Manual de la BIOS'', Marzo 2001

\end{thebibliography}
\end{document}
