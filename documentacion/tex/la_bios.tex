\section{La BIOS}\label{sec:bios}

	BIOS significa Basic Input/Output System, o lo que es lo mismo, Sistema
	Básico de Entrada y Sálida. La BIOS es un programa informático (es decir,
	Software). \\
	Este programa es el que se encarga de comprobar el hardware
	instalado en el sistema, manipular periféricos y dispositivos a bajo nivel
	y cargar el sistema de arranque que permite iniciar el sistema operativo. \\
	{\bf En resumen, es lo que permite que el ordenador arranque correctamente en
	primera instancia.}

	%\subsection{Funciones}\label{sub:funciones}
		%%A power-on self-test (POST) for all of the different hardware
		%%components in the system to make sure everything is working properly 
		
		%%Activating other BIOS chips on different cards installed in the
		%%computer - For example, SCSI and graphics cards often have their own
		%%BIOS chips.  
		
		%%Providing a set of low-level routines that the operating system uses to
		%%interface to different hardware devices - It is these routines that
		%%give the BIOS its name. They manage things like the keyboard, the
		%%screen, and the serial and parallel ports, especially when the computer
		%%is booting.  
		
		%%Managing a collection of settings for the hard disks, clock, etc.	

		%\begin{enumerate}
			%\item Realizar el POST
			%\item Detectar e instalar otros dispositivos, como disco duro,
				%accelerador grafico
			%\item 
		%\end{enumerate}
			
	\subsection{Marcas de BIOS}\label{sub:marcas de bios}
		Para mencionar algunas:
		\begin{enumerate}
			\item Award
			\item Phoenix
			\item American Megatrends Inc. (AMI)
		\end{enumerate}
	
	\subsection{Tipos de BIOS}\label{sub:chips bios}
		
		\begin{description}
			\item[ROM] Sólo se puede grabar en el momento que se fabrica el
				chip. La información que contiene no se puede alterar.
			\item[EEPROM] Estos chips se pueden grabar con luz ultravioleta. En
				la parte superior del chip se puede apreciar una especie de
				ventanilla transparente, que suele estar tapada con una
				pegatina. Estas BIOS se encuentra principalmente en 286 y 386.
				\cite{EEPROM}
			\item[Flash BIOS] Son los más utilizados en la actualidad. Estos
				chips se pueden grabar mediante impulsos eléctricos por lo que
				el propietario del ordenador la puede actualizar con un
				programa.
		\end{description}

	\subsection{Errores en pantalla}\label{sub:errores en pantalla}
	
	Los siguientes errores de pantalla son errores generales o bien, no
	dependen de marca y modelo de la BIOS:

		\subsubsection{BIOS ROM checksum error – system halted}

		El código de control de la BIOS es incorrecto, lo que indica que puede
		estar corrupta. En caso de reiniciar y repetir el mensaje, tendremos
		que reemplazar al BIOS.

		\subsubsection{CMOS battery failed}

		La pila de la placa base que alimenta la memoria CMOS ha dejado de
		suministrar corriente. Es necesario cambiar la pila inmediatamente.

		\subsubsection{CMOS checksum error – Defaults loaded}

		El código de control de la CMOS no es correcto, por lo que se
		procede a cargar los parámetros de la BIOS por defecto. Este error
		se produce porque la información almacenada en la CMOS es
		incorrecta, lo que puede indicar que la pila está empezando a
		fallar.

		\subsubsection{Display switch is set incorrectly}

		El tipo de pantalla especificada en la BIOS es incorrecta. Esto puede
		ocurrir si hemos seleccionado la existencia de un adaptador monocromo
		cuando tenemos uno en color, o al contrario. Bastará con poner bien
		este parámetro para solucionar el problema.

		\subsubsection{Floppy disk(s) Fail} 
		
		(code 40/38/48 dependiendo de la antigüedad de la bios)

		Disquetera mal conectada, verificamos todos los cables de conexión.

		\subsubsection{Hard disk install failure}

		La BIOS no es capaz de inicializar o encontrar el disco duro de
		manera correcta. Debemos estar seguros de que todos de que todos
		los discos se encuentren bien conectados y correctamente
		configurados.

		\subsubsection{Keyboard error or no keyboard present}

		No es posible inicializar el teclado. Puede ser debido a que no se
		encuentre conectado, este estropeado e incluso porque mantenemos
		pulsada alguna tecla durante el proceso de arranque.

		\subsubsection{Keyboard error is locked out – Unlock the key}

		Este mensaje solo aparece en muy pocas BIOS, cuando alguna tecla ha
		quedado presionada. 

		\subsubsection{Memory Test Fail}

		El chequeo de memoria RAM ha fallado debido probablemente, a
		errores en los módulos de memoria. En caso de que nos aparezca este
		mensaje, hemos de tener mucha precaución con el equipo, se puede
		volver inestable y tener prdidas de datos. Solución, comprobar que
		banco de memoria está mal, y sustituirlo inmediatamente. 

		\subsubsection{Override enabled – Defaults loaded}

		Si el sistema no puede iniciarse con los valores almacenados en la
		CMOS, la BIOS puede optar por sustituir estos por otros genricos
		diseñados para que todo funcione de manera estable, aunque sin
		obtener las mayores prestaciones. 

		\subsubsection{Primary master hard diskfail}

		El proceso de arranque ha detectado un fallo al iniciar el disco
		colocado como maestro en el controlador IDE primario. Para
		solucionar comprobaremos las conexiones del disco y la
		configuración de la BIOS. 

	\newpage
