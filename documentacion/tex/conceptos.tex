\section{Conceptos}\label{sec:conceptos}

	\subsection{La RAM}{\label{sec:conceptos/RAM}}

		La memoría de acceso aleatorio (en {\bfseries inglés Random Access Memory} cuyo
		acrónimo es {\bfseries RAM}) es la memoría donde el procesador recibe las
		instrucciones y guarda los resultados. Es el área de trabajo paara la mayor
		parte del software de un computador.

	\subsection{La ROM}{\label{sec:conceptos/ROM}}

		Memoría de solo lectura ({\bfseries Read-only memory}) es una
		clase de medio de almacenamiento usado en los ordenadores y otros dispositivos
		electrónicos. Los
		datos en la ROM no se pueden modificar al menos no de una manera rápida o fácil
		y se usa principalmente para contener el Firmware.

	\subsection{El Firmware}{\label{sec:conceptos/firmware}}

		Firmware o programación en firme, es un bloque de instrucciones de
		programa para propositos específicos grabada en una memoría de tipo no volátil
		( ROM, EPROM, Flash, ...) que establece la lógica de más bajo nivel que
		controla los circuitos electrónicos de un dispositivo de cualquier tipo.

	\subsection{El CMOS}{\label{sec:conceptos/CMOS}}

		{\bf CMOS} es la contracción de {\em Complementary Metal Oxide Semiconductor}, o
		Tecnología Metal Óxido Semiconductor Complementario. \\
		Es una tecnología utilizada para fabricar circuitos integrados (chips), que
		para el caso es un memoria del tipo ROM, en la actualidad FLASH ROM (se
		borran electricamente) donde se guarda información básica del equipo como el
		BIOS y el SETUP. \\ 
		{\bf Esta memoria tiene una porción que funciona como RAM, donde se guarda la
		configuración que el usuario le da al equipo y algunas configuraciones
		establecidas de fábrica.} \\ 
		Otra información que se guarda en el y que se accede mediante el SETUP
		es la fecha y la hora del sistema, y para que la misma se mantenga
		actualizada debe estar siempre alimentada, es por esto que en toda
		computadora encontraremos una pila para alimentar dicho circuito.

		\subsubsection{El RAM-CMOS}\label{sub:el ram-cmos}
			RAM-CMOS es un tipo de memoria en que se guardan los datos que se
			pueden configurar del BIOS y contiene información básica sobre algunos
			recursos del sistema que son susceptibles de ser modificados como el
			disco duro, el tipo de disco flexible, etc. \\
			Esta información es almacenada en una RAM, de 64 bytes de capacidad,
			con tecnología CMOS, que le proporciona el bajo consumo necesario para
			ser alimentada por una pila que se encuentra en la placa base y que
			debe durar años, al ser necesario que este alimentada constantemente,
			incluso cuando el ordenador se encuentra apagado. Para ello
			antiguamente se usaba una batería recargable que se cargaba cuando el
			ordenador se encendía. \\
			Mas modernamente se ha sustituido por una pila desechable de litio
			(generalmente modelo CR-2032) y que dura de 2 a 5 años.

	\subsection{El BIOS}{\label{sec:conceptos/BIOS}}

		BIOS es la contracción de Basic Input Output System, o Sistema Básico de
		Entrada – Salida. \\
		Es un programa muy básico, normalmente programado en lenguaje ensamblador,
		cuya misión es la de arrancar o posibilitar el ``Booteo'' de la computadora.
		A pesar de tratarse de un programa sumamente básico resulta totalmente
		indispensable, ya que sin el BIOS sería imposible que una computadora pudiera
		iniciar.

	\subsection{CMOS Setup}\label{sub:cmos setup}
	
		Basicamente el CMOS Setup es el programa que puede modificar todos los parametros
		guardada en la CMOS de forma visual y tangible (con el teclado). \\
		El CMOS Setup es una serie de instrucciones generalmente escrito en
		lenguaje Ensemblador que se lee de la BIOS al arrancar la computadora. \\
		Se suelen usar los terminos CMOS Setup y BIOS Setup como sinonimos para
		indicar lo mismo, pero son dos diferentes circuitos en la placa base. \\

	\subsection{LBA}\label{sec:lba}

		LBA (siglas de logical block addressing, dirección lógica de bloques)
		es un método muy común usado para especificar la localización de los
		bloques de datos en los sistemas de almacenamiento, principalmente
		secundario, del ordenador. El término LBA puede referirse también a la
		dirección del bloque al que enlaza.  Los bloques lógicos en los
		ordenadores modernos son normalmente de 512 o 1024 bytes cada uno.

	\newpage
