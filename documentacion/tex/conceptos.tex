\section{Conceptos}\label{sec:conceptos}

	\subsection{La RAM}{\label{sec:conceptos/RAM}}

		La memoría de acceso aleatorio (en {\bfseries inglés Random Access Memory} cuyo
		acrónimo es {\bfseries RAM}) es la memoría donde el procesador recibe las
		instrucciones y guarda los resultados. Es el área de trabajo para la mayor
		parte del software de un computador.

	\subsection{La ROM}{\label{sec:conceptorom}}

		Memoría de solo lectura ({\bfseries Read-only memory}) es una
		clase de medio de almacenamiento usado en los ordenadores y otros dispositivos
		electrónicos. Los
		datos en la ROM no se pueden modificar al menos no de una manera rápida o fácil
		y se usa principalmente para contener el Firmware.

	\subsection{El Firmware}{\label{sec:conceptos/firmware}}

		Firmware o programación en firme, es un bloque de instrucciones de
		programa para propositos específicos grabada en una memoría de tipo no volátil
		( ROM, EPROM, Flash, ...) que establece la lógica de más bajo nivel que
		controla los circuitos electrónicos de un dispositivo de cualquier tipo.

	\subsection{La CMOS}{\label{sec:conceptos/CMOS}}

		{\bf CMOS} es la abreviación que proviende el ingles de {\em
		Complementary Metal Oxide Semiconductor}, o Tecnología Metal Óxido
		Semiconductor Complementario. \\ 
		Es una tecnología utilizada para fabricar circuitos integrados (chips),
		que para el caso es un memoria del tipo ROM, en la actualidad FLASH ROM
		(se borran electricamente) donde se guarda información básica del
		equipo como la BIOS y el SETUP.
		\\ Esta memoria tiene una porción que funciona como RAM, donde se
		guarda la configuración que el usuario le da al equipo y algunas
		configuraciones establecidas de fábrica.{\bf La RAM-CMOS} \\ 

		\subsubsection{El RAM-CMOS}\label{sub:el ram-cmos}

			RAM-CMOS es un tipo de memoria en que se guardan los datos que se
			pueden configurar de la BIOS y contiene información básica sobre algunos
			recursos del sistema que son susceptibles de ser modificados como el
			disco duro, el tipo de disco flexible, etc. \\
			Esta información es almacenada en una RAM, de 64 Bytes o en las
			computadores de la nueva generación de 512 bytes de capacidad, con
			tecnología CMOS, que le proporciona el bajo consumo necesario para
			ser alimentada por una pila que se encuentra en la placa base y que
			debe durar años, al ser necesario que este alimentada constantemente,
			incluso cuando el ordenador se encuentra apagado. Para ello
			antiguamente se usaba una batería recargable que se cargaba cuando el
			ordenador se encendía. \\
			Mas modernamente se ha sustituido por una pila desechable de litio
			(generalmente modelo CR-2032) y que dura de 2 a 5 años.

	\subsection{La BIOS}\label{sec:conceptobios}

		BIOS es la contracción de Basic Input Output System, o Sistema Básico de
		Entrada – Salida. \\
		Es un programa muy básico, normalmente programado en lenguaje ensamblador,
		cuya misión es la de arrancar o posibilitar el ``Booteo'' de la computadora.
		A pesar de tratarse de un programa sumamente básico resulta totalmente
		indispensable, ya que sin la BIOS sería imposible que una computadora pudiera
		iniciar.

	\subsection{POST}\label{sec:post}

		(Power-On Self Test, Test Automático de Encendido), un pequeño test que
		comprueba que todo esté conectado correctamente y que no haya ningún
		problema en los dispositivos. Si todo está correcto, dará paso a cargar
		el sistema operativo, en caso contrario, reportará un mensaje de
		error o nos informará de algún fallo mediante una serie de pitidos o
		por voz si nuestra placa base incorpora esta funcionalidad.

	\subsection{CMOS Setup}\label{sub:cmos setup}
	
		Se llama así al programa que nos permite acceder a los datos de la CMOS
		y que por eso también se suele denominar CMOS Setup. Este programa
		suele activarse al pulsar cierta/s tecla/s durante el arranque del
		ordenador. Usamos este programa para consultar y/o modificar la
		información de la CMOS (cuántos discos duros y de qué características;
		la fecha y hora, etc). \\
		Lógicamente, este programa SETUP está ``archivado'' (guardado) en
		alguna parte dentro del ordenador y debe funcionar incluso cuando no
		hay disco duro o cuando todavía no se ha reconocido el disco duro: el
		Setup está guardado dentro de la ROM-BIOS.  

	\newpage
