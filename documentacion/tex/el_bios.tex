\section{El BIOS}{\label{sec:bios}}

	Es indispensable conocer el concepto y las funciones que cumple el BIOS antes de tratar el CMOS Setup.

	El concepto del BIOS:
	\begin{quotation}

		{\em BIOS es la contracción de Basic Input Output System, o Sistema Básico de
		Entrada – Salida. 
		Es un programa muy básico, normalmente programado en lenguaje ensamblador,
		cuya misión es la de arrancar o posibilitar el “Booteo” de la computadora.
		A pesar de tratarse de un programa sumamente básico resulta totalmente
		indispensable, ya que sin el BIOS sería imposible que una computadora pudiera
		iniciar.}

	\end{quotation}
			
	\subsection{Marcas de BIOS}\label{sub:marcas de bios}
		Para mencionar algunas:
		\begin{enumerate}
			\item Award
			\item Phoenix
			\item American Megatrends Inc. (AMI)
		\end{enumerate}
	
	\subsection{Tipos de BIOS}\label{sub:chips bios}
		
		\begin{description}
			\item[ROM] Sólo se puede grabar en el momento que se fabrica el
				chip. La información que contiene no se puede alterar.
			\item[EEPROM] Estos chips se pueden grabar con luz ultravioleta. En
				la parte superior del chip se puede apreciar una especie de
				ventanilla transparente, que suele estar tapada con una
				pegatina. Estas BIOS se encuentra principalmente en 286 y 386.
				\cite{EEPROM}
			\item[Flash BIOS] Son los más utilizados en la actualidad. Estos
				chips se pueden grabar mediante impulsos eléctricos por lo que
				el propietario del ordenador la puede actualizar con un
				programa.
		\end{description}

		\newpage
		%\subsection{Solución de problemas}{\label{sec:bios/solucion-de-problemas}}
			%\subsubsection{Pitidos de la BIOS (Award)}{\label{sec:bios/tonos-de-la-bios}}

			%En esta seccion cubriremos los significados de los pitidos de los BIOS {\em Award}. \\
			%En la mayoría de los pitidos se les acompaña un mensaje de error. 

				%\paragraph{Tono ininterrumpido}:
				
				%Fallo en el suministro elctrico. Revisamos las conexiones y la fuente
				%de alimentación.  Tonos cortos constantes: Sobrecarga elctrica, chips
				%defectuosos, placa mal.

				%\paragraph{1 largo}:

				%Si aparece esto en la pantalla “RAM Refresh Failure”, significa que los
				%diferentes componentes encargados del refresco de la memoria RAM fallan
				%o no están presentes. Cambiar de banco la memoria y comprobar los
				%jumpers de buses. 

				%\paragraph{1 largo y 1 corto}: 

				%El código de la BIOS esta corrupto o defectuoso, probaremos a flasear o
				%reemplazamos el chip de la BIOS sino podemos cambiamos de placa. 

				%\paragraph{1 largo y dos cortos}:
				
				%No da señal de imagen, se trata de que nuestra tarjeta de vídeo esta
				%estropeada, probaremos a pincharla en otro slot o probaremos otra
				%tarjeta gráfica. 

				%\paragraph{1 largo y 2 cortos}:

				%Si aparece por pantalla este mensaje: “No video card found”, este error
				%solo es aplicable a placas base con tarjetas de vídeo integradas. Fallo
				%en la tarjeta gráfica, probaremos a deshabilitarla y pincharemos una
				%nueva en cualquier slot libre o cambiaremos la placa madre. 

				%\paragraph{1 largo y 3 cortos}:

				%Si aparece este mensaje por pantalla “No monitor connected” Idem que el
				%anterior. 

				%\paragraph{1 largo y varios cortos}: 

				%Mensaje de error. “Video related failure”. Lo mismo que antes. Cada
				%fabricante implanta un código de error según el tipo de tarjeta de
				%video y los parámetros de cada BIOS.

				%\paragraph{2 largos y 1 corto}:
				
				%Fallo en la sincronización de las imágenes. Cargaremos por defecto los
				%valores de la BIOS e intentaremos reiniciar. Si persiste nuestra
				%tarjeta gráfica o placa madre están estropeadas. 

				%\paragraph{2 cortos}:
				
				%Vemos en la pantalla este error: “Parity Error”. Se trata de un error
				%en la configuración de la BIOS al no soportar la paridad de memoria, la
				%deshabilitamos en al BIOS. 

				%\paragraph{3 cortos}:
				
				%Vemos en la pantalla este error. Base 64 Kb “Memory Failure”, significa
				%que la BIOS al intentar leer los primeros 64Kbytes de memoria RAM
				%dieron error. Cambiamos la RAM instalada por otra. 

				%\paragraph{4 cortos}:
				
				%Mensaje de error; “Timer not operational”. El reloj de la propia placa
				%base esta estropeado, no hay más solución que cambiar la placa. No
				%confundir con “CMOS cheksum error” una cosa es la pila y otra el
				%contador o reloj de la placa base. 

				%\paragraph{5 cortos}:
				
				%Mensaje por pantalla ``Processor Error'' significa que la CPU ha generado
				%un error porque el procesador o la memoria de vídeo están bloqueados. 

				%\paragraph{6 cortos}:

				%Mensaje de error: ``8042 - Gate A20 Failure'', muy mítico este error.
				%El controlador o procesador del teclado (8042) puede estar en mal
				%estado. 
				%La BIOS no puede conmutar en modo protegido. Este error se
				%suele dar cuando se conecta/desconecta el teclado con el ordenador
				%encendido. 

				%\paragraph{7 cortos}:
				
				%Mensaje de error: “Processor Exception / Interrupt Error”
				%Descripción. La CPU ha generado una interrupción excepcional o
				%el modo virtual del procesador está activo. Procesador a punto
				%de morirse. 

				%\paragraph{8 cortos}:
				
				%Mensaje de error: ``Display Memory Read / Write error''. La tarjeta de video esta estropeada, procedemos a cambiarla. 

				%\paragraph{9 cortos}:
				
				%Mensaje de error: ``ROM Checksum Error''; el valor del checksum (conteo de la memoria) de la RA

			%\newpage

	\subsection{Errores en pantalla}\label{sub:errores en pantalla}
	
	Los siguientes errores de pantalla son errores generales o bien, no
	dependen de marca y modelo de la BIOS:

		\subsubsection{BIOS ROM checksum error – system halted}

		El código de control de la BIOS es incorrecto, lo que indica que puede
		estar corrupta. En caso de reiniciar y repetir el mensaje, tendremos
		que reemplazar al BIOS.

		\subsubsection{CMOS battery failed}

		La pila de la placa base que alimenta la memoria CMOS ha dejado de
		suministrar corriente. Es necesario cambiar la pila inmediatamente.

		\subsubsection{CMOS checksum error – Defaults loaded}

		El código de control de la CMOS no es correcto, por lo que se
		procede a cargar los parámetros de la BIOS por defecto. Este error
		se produce porque la información almacenada en la CMOS es
		incorrecta, lo que puede indicar que la pila está empezando a
		fallar.

		\subsubsection{Display switch is set incorrectly}

		El tipo de pantalla especificada en la BIOS es incorrecta. Esto puede
		ocurrir si hemos seleccionado la existencia de un adaptador monocromo
		cuando tenemos uno en color, o al contrario. Bastará con poner bien
		este parámetro para solucionar el problema.

		\subsubsection{Floppy disk(s) Fail} 
		
		(code 40/38/48 dependiendo de la antigüedad de la bios)

		Disquetera mal conectada, verificamos todos los cables de conexión.

		\subsubsection{Hard disk install failure}

		La BIOS no es capaz de inicializar o encontrar el disco duro de
		manera correcta. Debemos estar seguros de que todos de que todos
		los discos se encuentren bien conectados y correctamente
		configurados.

		\subsubsection{Keyboard error or no keyboard present}

		No es posible inicializar el teclado. Puede ser debido a que no se
		encuentre conectado, este estropeado e incluso porque mantenemos
		pulsada alguna tecla durante el proceso de arranque.

		\subsubsection{Keyboard error is locked out – Unlock the key}

		Este mensaje solo aparece en muy pocas BIOS, cuando alguna tecla ha
		quedado presionada. 

		\subsubsection{Memory Test Fail}

		El chequeo de memoria RAM ha fallado debido probablemente, a
		errores en los módulos de memoria. En caso de que nos aparezca este
		mensaje, hemos de tener mucha precaución con el equipo, se puede
		volver inestable y tener prdidas de datos. Solución, comprobar que
		banco de memoria está mal, y sustituirlo inmediatamente. 

		\subsubsection{Override enabled – Defaults loaded}

		Si el sistema no puede iniciarse con los valores almacenados en la
		CMOS, la BIOS puede optar por sustituir estos por otros genricos
		diseñados para que todo funcione de manera estable, aunque sin
		obtener las mayores prestaciones. 

		\subsubsection{Primary master hard diskfail}

		El proceso de arranque ha detectado un fallo al iniciar el disco
		colocado como maestro en el controlador IDE primario. Para
		solucionar comprobaremos las conexiones del disco y la
		configuración de la BIOS. 
	

	\newpage
	\newpage
