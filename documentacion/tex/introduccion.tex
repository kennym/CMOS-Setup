\section{Introducción}\label{sec:introduccion}

Cuando encendemos la computadora, el sistema operativo se encuentra o bien, en
el disco duro o bien en un disquete; sin embargo, si se supone que es el
sistema operativo el que debe dar soporte para estos dispositivos, {\em ¿cómo podría
hacerlo si aún no está cargado en memoria?}

Lo que es más:
\begin{itemize}
	\item ¿Cómo sabe la computadora que tiene un disco duro (o varios)?
	\item ¿Y la disquetera?  
	\item ¿Cómo y dónde guarda esos datos, junto con el tipo de memoria y cache?
	\item ¿O algo tan sencillo pero importante como la fecha y la hora? 
\end{itemize}

Por ende, la unidad y programa que se encarga de todo esto es la BIOS.
Resulta evidente que la BIOS debe poderse modificar para alterar estos datos
(al añadir un disco duro o cambiar al horario de verano, por ejemplo); por ello
las BIOS se implementan en memoria. \\
Pero además debe mantenerse cuando apaguemos nuestra pc, porque no tendría
sentido tener que introducir todos los datos en cada arranque; por eso se usan
memorias especiales, que no se borran al apagar el ordenador: memorias tipo
CMOS (no volatil), por lo que muchas veces el programa que modifica la BIOS se
denomina {\bf CMOS Setup}.

\newpage
